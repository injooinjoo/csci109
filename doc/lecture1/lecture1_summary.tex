\documentclass[12pt,a4paper]{article}
\usepackage{fontspec}
\usepackage{amsmath,amssymb}
\usepackage{graphicx}
\usepackage{hyperref}
\usepackage{geometry}
\geometry{margin=2.5cm}

% Korean font setup for XeLaTeX
\setmainfont{AppleGothic}

\title{\textbf{CS109A: Introduction to Data Science} \\ Lecture 1 - 강의 요약}
\author{Pavlos Protopapas, Kevin Rader, Chris Gumb \\ Harvard University}
\date{Fall 2024}

\begin{document}

\maketitle

\tableofcontents
\newpage

\section{강의 개요 (Course Overview)}

CS109A는 데이터 과학 입문 과정으로, 통계학과 컴퓨터 과학의 교차점에서 현대 데이터 분석 기법을 다룹니다. 본 강의는 STAT109A, AC209A, CSCIE-109A와 동일한 과정입니다.

\subsection{강의 목표 (Course Goals)}

이 과정을 통해 학생들은 다음을 배우게 됩니다:
\begin{itemize}
    \item 데이터로부터 유용한 예측과 인사이트를 도출하는 방법
    \item 통계적 모델링과 머신러닝 기법의 기초
    \item 실제 데이터를 다루고 분석하는 실용적 기술
    \item Python을 활용한 데이터 과학 도구 사용법
\end{itemize}

\subsection{강사진 (Instructors)}

\textbf{Pavlos Protopapas} (수석 강사):
\begin{itemize}
    \item Scientific Program Director, Institute for Applied Computational Science
    \item 천체물리학 배경, 데이터 과학 및 머신러닝 전문가
    \item 18년간 하버드에서 데이터 과학 교육
\end{itemize}

\textbf{Kevin Rader} (공동 강사):
\begin{itemize}
    \item Senior Preceptor, Statistics Department
    \item 통계학 박사, 데이터 과학 교육 전문
\end{itemize}

\textbf{Chris Gumb} (공동 강사):
\begin{itemize}
    \item Preceptor, IACS
    \item 소프트웨어 엔지니어링 배경
\end{itemize}

\section{데이터 과학의 역사와 발전 (History and Evolution of Data Science)}

\subsection{과학적 방법론의 변화 (Evolution of Scientific Methods)}

데이터 과학은 다음과 같은 네 가지 패러다임을 거쳐 발전했습니다:

\begin{enumerate}
    \item \textbf{경험적 관찰 (Empirical Observation)}: 자연 현상의 직접 관찰
    \item \textbf{이론적 모델 (Theoretical Models)}: 수학적 방정식을 통한 현상 설명
    \item \textbf{컴퓨터 시뮬레이션 (Computational Simulation)}: 복잡한 시스템의 모델링
    \item \textbf{데이터 과학/머신러닝 (Data Science/ML)}: 데이터로부터 패턴 발견
\end{enumerate}

\subsection{데이터 과학의 응용 분야 (Applications)}

현대 데이터 과학은 다양한 분야에서 활용됩니다:

\begin{itemize}
    \item \textbf{의료 진단 (Medical Diagnosis)}: 질병 예측 및 진단
    \item \textbf{생성형 AI (Generative AI)}: 텍스트, 이미지, 음성 생성
    \item \textbf{신약 개발 (Drug Discovery)}: 분자 구조 예측 및 최적화
    \item \textbf{교통 시스템 (Transportation)}: 자율주행 및 경로 최적화
    \item \textbf{기후 과학 (Climate Science)}: 기후 변화 예측 및 분석
\end{itemize}

\section{데이터 과학 프로세스 (The Data Science Process)}

데이터 과학 프로젝트는 다음 5단계로 구성됩니다:

\subsection{1단계: 질문 정의 (Ask an Interesting Question)}

\textbf{핵심 개념:} 명확하고 측정 가능한 연구 질문을 설정합니다.

\textbf{예시:}
\begin{itemize}
    \item 어떤 요인이 주택 가격에 영향을 미치는가?
    \item 고객 이탈을 예측할 수 있는가?
    \item 이미지에서 특정 객체를 분류할 수 있는가?
\end{itemize}

\subsection{2단계: 데이터 수집 (Get the Data)}

\textbf{데이터 소스:}
\begin{itemize}
    \item 웹 스크래핑 (Web Scraping)
    \item API를 통한 데이터 수집
    \item 공개 데이터셋 (Public Datasets)
    \item 실험 및 설문조사
\end{itemize}

\textbf{도구:} Python의 BeautifulSoup, Selenium, pandas 등

\subsection{3단계: 데이터 탐색 (Explore the Data)}

\textbf{탐색적 데이터 분석 (Exploratory Data Analysis, EDA):}
\begin{itemize}
    \item 데이터 시각화 (Visualization)
    \item 기술 통계량 계산 (Descriptive Statistics)
    \item 이상치 탐지 (Outlier Detection)
    \item 패턴 및 관계 파악
\end{itemize}

\textbf{주요 통계량:}

평균 (Mean):
\[
\bar{x} = \frac{1}{n}\sum_{i=1}^{n} x_i
\]

분산 (Variance):
\[
\sigma^2 = \frac{1}{n}\sum_{i=1}^{n} (x_i - \bar{x})^2
\]

표준편차 (Standard Deviation):
\[
\sigma = \sqrt{\sigma^2}
\]

\subsection{4단계: 모델링 (Model the Data)}

\textbf{모델 유형:}

\begin{itemize}
    \item \textbf{회귀 모델 (Regression Models)}: 연속적인 값 예측
    \begin{itemize}
        \item 선형 회귀 (Linear Regression)
        \item 다항 회귀 (Polynomial Regression)
        \item 정규화 회귀 (Ridge, Lasso)
    \end{itemize}

    \item \textbf{분류 모델 (Classification Models)}: 범주형 결과 예측
    \begin{itemize}
        \item 로지스틱 회귀 (Logistic Regression)
        \item 의사결정 트리 (Decision Trees)
        \item 랜덤 포레스트 (Random Forests)
    \end{itemize}

    \item \textbf{베이지안 모델 (Bayesian Models)}: 확률적 추론
    \item \textbf{신경망 (Neural Networks)}: 딥러닝 모델
\end{itemize}

\textbf{선형 회귀 예시:}

예측 함수:
\[
\hat{y} = \beta_0 + \beta_1 x_1 + \beta_2 x_2 + \cdots + \beta_p x_p
\]

손실 함수 (평균 제곱 오차):
\[
MSE = \frac{1}{n}\sum_{i=1}^{n} (y_i - \hat{y}_i)^2
\]

\subsection{5단계: 결과 전달 (Communicate and Visualize Results)}

\textbf{핵심 요소:}
\begin{itemize}
    \item 명확한 시각화 (Clear Visualizations)
    \item 결과의 해석 (Interpretation)
    \item 불확실성 전달 (Uncertainty Communication)
    \item 실행 가능한 인사이트 (Actionable Insights)
\end{itemize}

\section{강의 구조 및 일정 (Course Structure and Schedule)}

\subsection{주간 일정 (Weekly Schedule)}

\begin{table}[h]
\centering
\begin{tabular}{|c|l|p{8cm}|}
\hline
\textbf{주차} & \textbf{주제} & \textbf{내용} \\
\hline
1 & Introduction & 데이터 과학 개요, 데이터 타입, 시각화 \\
\hline
2 & Data Collection & 웹 스크래핑, pandas 기초 \\
\hline
3 & EDA & 탐색적 데이터 분석, 통계량 \\
\hline
4 & Linear Regression & 선형 회귀 모델, 최소제곱법 \\
\hline
5 & Multiple Regression & 다중 회귀, 모델 선택 \\
\hline
6 & Regularization & Ridge, Lasso, Elastic Net \\
\hline
7 & Classification & 로지스틱 회귀, 분류 평가 지표 \\
\hline
8 & Midterm Review & 중간고사 준비 및 복습 \\
\hline
9 & Tree-based Models & 의사결정 트리, 랜덤 포레스트 \\
\hline
10 & Boosting & Gradient Boosting, XGBoost \\
\hline
11 & Neural Networks & 신경망 기초, 역전파 \\
\hline
12 & Deep Learning & CNN, RNN 개요 \\
\hline
13 & Bayesian Modeling & 베이지안 추론, MCMC \\
\hline
14 & Review \& Projects & 최종 복습 및 프로젝트 발표 \\
\hline
\end{tabular}
\caption{CS109A 주간 강의 일정}
\end{table}

\subsection{강의 형식 (Course Format)}

\begin{itemize}
    \item \textbf{강의 (Lectures)}: 주 2회, 각 90분
    \item \textbf{실습 (Labs)}: 주 1회, 실습 문제 풀이
    \item \textbf{오피스 아워 (Office Hours)}: Teaching Fellows와의 질의응답
    \item \textbf{온라인 자료}: Ed Discussion, Canvas
\end{itemize}

\section{성적 평가 (Grading)}

\subsection{평가 구성 (Grade Components)}

\begin{table}[h]
\centering
\begin{tabular}{|l|c|l|}
\hline
\textbf{항목} & \textbf{비율} & \textbf{세부사항} \\
\hline
숙제 (Homework) & 30\% & 5개 과제, 각 6\% \\
\hline
퀴즈 (Quizzes) & 10\% & 2개 퀴즈, 각 5\% \\
\hline
중간고사 (Midterm) & 18\% & 1회 \\
\hline
기말고사 (Final) & 22\% & 1회 \\
\hline
프로젝트 (Project) & 20\% & 팀 프로젝트 \\
\hline
\textbf{총계} & \textbf{100\%} & \\
\hline
\end{tabular}
\caption{성적 평가 구성}
\end{table}

\subsection{숙제 정책 (Homework Policy)}

\textbf{제출 규정:}
\begin{itemize}
    \item 총 5개의 숙제 (각 6\%)
    \item Jupyter Notebook 형식으로 제출
    \item Late Days: 학기당 3일 사용 가능
    \item 협업 허용, 단 각자 코드 작성 필수
\end{itemize}

\textbf{Late Day 정책:}
\begin{itemize}
    \item 학기당 총 3일의 지각 허용일 제공
    \item 1일 단위로 사용 (부분 사용 불가)
    \item 미리 알릴 필요 없음, 자동 적용
    \item 모두 소진 후에는 하루당 10\% 감점
\end{itemize}

\subsection{출석 정책 (Attendance Policy)}

\textbf{중요:} 출석은 성적 자격요건입니다!

\begin{table}[h]
\centering
\begin{tabular}{|l|c|}
\hline
\textbf{출석률} & \textbf{최대 성적} \\
\hline
90\% 이상 & A (제한 없음) \\
\hline
80-89\% & B+ \\
\hline
70-79\% & C+ \\
\hline
70\% 미만 & D+ \\
\hline
\end{tabular}
\caption{출석률에 따른 성적 상한}
\end{table}

\textbf{출석 인정 방법:}
\begin{itemize}
    \item 강의 참석
    \item 강의 녹화 시청 (퀴즈 응시)
    \item 실습 세션 참여
\end{itemize}

\section{프로젝트 (Course Project)}

\subsection{프로젝트 구조}

\textbf{팀 구성:}
\begin{itemize}
    \item 3-4명으로 구성된 팀
    \item 자유롭게 주제 선택
    \item 실제 데이터셋 사용
\end{itemize}

\textbf{프로젝트 단계:}
\begin{enumerate}
    \item \textbf{제안서 (Proposal)}: 연구 질문 및 데이터 설명
    \item \textbf{중간 보고서 (Milestone)}: 초기 분석 결과
    \item \textbf{최종 보고서 (Final Report)}: 완전한 분석 및 결과
    \item \textbf{발표 (Presentation)}: 5-10분 팀 발표
\end{enumerate}

\subsection{평가 기준}

프로젝트는 다음 기준으로 평가됩니다:
\begin{itemize}
    \item \textbf{질문의 명확성}: 연구 질문이 구체적이고 측정 가능한가?
    \item \textbf{데이터 분석}: EDA가 충분히 수행되었는가?
    \item \textbf{모델링}: 적절한 모델을 선택하고 정당화했는가?
    \item \textbf{결과 해석}: 결과를 명확하게 전달했는가?
    \item \textbf{코드 품질}: 코드가 깔끔하고 재현 가능한가?
\end{itemize}

\section{필수 도구 및 기술 (Required Tools and Skills)}

\subsection{프로그래밍 언어}

\textbf{Python 3.x} (필수)
\begin{itemize}
    \item NumPy: 수치 계산
    \item pandas: 데이터 조작
    \item matplotlib/seaborn: 시각화
    \item scikit-learn: 머신러닝
    \item statsmodels: 통계 모델링
\end{itemize}

\subsection{개발 환경}

\textbf{Jupyter Notebook/JupyterLab:}
\begin{itemize}
    \item 대화형 코드 실행
    \item 시각화 통합
    \item Markdown 문서화
\end{itemize}

\textbf{설치 방법:}
\begin{verbatim}
# Anaconda 설치 권장
conda install jupyter
conda install numpy pandas matplotlib seaborn
conda install scikit-learn statsmodels
\end{verbatim}

\subsection{버전 관리}

\textbf{Git/GitHub:}
\begin{itemize}
    \item 코드 버전 관리
    \item 팀 협업
    \item 포트폴리오 구축
\end{itemize}

\section{데이터 타입 및 시각화 기초 (Data Types and Visualization Basics)}

\subsection{데이터 타입 분류}

\subsubsection{양적 데이터 (Quantitative Data)}

\textbf{1. 연속형 (Continuous):}
\begin{itemize}
    \item 정의: 무한한 값을 가질 수 있는 데이터
    \item 예시: 키, 몸무게, 온도, 시간
    \item 특징: 소수점 값 가능
\end{itemize}

\textbf{2. 이산형 (Discrete):}
\begin{itemize}
    \item 정의: 셀 수 있는 정수 값
    \item 예시: 학생 수, 판매량, 클릭 수
    \item 특징: 정수로만 표현
\end{itemize}

\subsubsection{질적 데이터 (Qualitative/Categorical Data)}

\textbf{1. 명목형 (Nominal):}
\begin{itemize}
    \item 정의: 순서가 없는 범주
    \item 예시: 색상, 성별, 국가
    \item 특징: 분류만 가능
\end{itemize}

\textbf{2. 순서형 (Ordinal):}
\begin{itemize}
    \item 정의: 순서가 있는 범주
    \item 예시: 학년, 만족도 (낮음/중간/높음), 순위
    \item 특징: 순서는 있으나 간격은 일정하지 않음
\end{itemize}

\subsection{기술 통계량 (Descriptive Statistics)}

\subsubsection{중심 경향성 (Measures of Central Tendency)}

\textbf{평균 (Mean):}
\[
\mu = \bar{x} = \frac{1}{n}\sum_{i=1}^{n} x_i
\]

특징: 모든 값을 고려, 이상치에 민감

\textbf{중앙값 (Median):}
\begin{itemize}
    \item 정의: 데이터를 정렬했을 때 중간 위치의 값
    \item 계산: $n$이 홀수면 $x_{(n+1)/2}$, 짝수면 $\frac{x_{n/2} + x_{n/2+1}}{2}$
    \item 특징: 이상치에 강건함
\end{itemize}

\textbf{최빈값 (Mode):}
\begin{itemize}
    \item 정의: 가장 빈번하게 나타나는 값
    \item 특징: 범주형 데이터에도 적용 가능
\end{itemize}

\subsubsection{산포도 (Measures of Spread)}

\textbf{분산 (Variance):}

모집단 분산:
\[
\sigma^2 = \frac{1}{N}\sum_{i=1}^{N} (x_i - \mu)^2
\]

표본 분산 (불편 추정량):
\[
s^2 = \frac{1}{n-1}\sum_{i=1}^{n} (x_i - \bar{x})^2
\]

\textbf{표준편차 (Standard Deviation):}
\[
\sigma = \sqrt{\sigma^2}, \quad s = \sqrt{s^2}
\]

\textbf{사분위수 범위 (Interquartile Range, IQR):}
\[
IQR = Q_3 - Q_1
\]
여기서 $Q_1$은 제1사분위수(25\%), $Q_3$는 제3사분위수(75\%)

\subsection{시각화 기법 (Visualization Techniques)}

\subsubsection{단변량 시각화 (Univariate)}

\textbf{1. 히스토그램 (Histogram):}
\begin{itemize}
    \item 용도: 연속형 데이터의 분포 확인
    \item 특징: 구간(bin)으로 나누어 빈도 표시
    \item Python 코드:
\begin{verbatim}
import matplotlib.pyplot as plt
plt.hist(data, bins=30, edgecolor='black')
plt.xlabel('Value')
plt.ylabel('Frequency')
\end{verbatim}
\end{itemize}

\textbf{2. 박스플롯 (Box Plot):}
\begin{itemize}
    \item 용도: 사분위수 및 이상치 확인
    \item 구성요소:
    \begin{itemize}
        \item 박스: $Q_1$부터 $Q_3$까지 (IQR)
        \item 중앙선: 중앙값 (Median)
        \item 수염: $Q_1 - 1.5 \times IQR$부터 $Q_3 + 1.5 \times IQR$까지
        \item 점: 이상치 (Outliers)
    \end{itemize}
\end{itemize}

\textbf{3. 막대 그래프 (Bar Chart):}
\begin{itemize}
    \item 용도: 범주형 데이터의 빈도 비교
    \item 특징: 각 범주별 개수 또는 비율 표시
\end{itemize}

\subsubsection{이변량 시각화 (Bivariate)}

\textbf{1. 산점도 (Scatter Plot):}
\begin{itemize}
    \item 용도: 두 연속형 변수 간의 관계 파악
    \item 패턴: 선형, 비선형, 상관관계 확인
\end{itemize}

\textbf{2. 라인 플롯 (Line Plot):}
\begin{itemize}
    \item 용도: 시간에 따른 변화 추적
    \item 특징: 순서가 중요한 데이터
\end{itemize}

\section{Teaching Fellows 및 지원 (Support Staff)}

\subsection{Teaching Fellows 팀}

강의를 지원하는 Teaching Fellows 팀:
\begin{itemize}
    \item 실습 세션 진행
    \item 오피스 아워 운영
    \item 숙제 및 프로젝트 피드백
    \item Ed Discussion 질문 답변
\end{itemize}

\textbf{역할:}
\begin{enumerate}
    \item \textbf{Head TF}: 전체 TF 팀 조율, 강의 지원 총괄
    \item \textbf{Lab TFs}: 실습 세션 진행, 실시간 코딩 도움
    \item \textbf{Grading TFs}: 과제 채점 및 피드백 제공
    \item \textbf{Discussion TFs}: 온라인 질문 답변
\end{enumerate}

\subsection{학습 지원 자료}

\textbf{1. Ed Discussion:}
\begin{itemize}
    \item 질문 및 답변 플랫폼
    \item 24시간 이내 응답 보장
    \item 학생 간 협력 학습
\end{itemize}

\textbf{2. Canvas:}
\begin{itemize}
    \item 강의 자료 업로드
    \item 과제 제출
    \item 성적 확인
\end{itemize}

\textbf{3. 오피스 아워:}
\begin{itemize}
    \item 주중 매일 운영
    \item 1:1 또는 소그룹 지도
    \item 예약 없이 방문 가능
\end{itemize}

\section{자주 묻는 질문 (FAQ)}

\subsection{강의 관련}

\textbf{Q1: Python 경험이 없어도 수강 가능한가요?}

A: 기본적인 프로그래밍 경험이 있다면 가능합니다. 첫 2주간 Python 기초를 다루며, 추가 자료도 제공됩니다.

\textbf{Q2: 통계학 배경이 필요한가요?}

A: 기초 통계(평균, 표준편차)를 알면 유리하지만 필수는 아닙니다. 필요한 통계 개념은 강의에서 다룹니다.

\textbf{Q3: 프로젝트 주제는 어떻게 정하나요?}

A: 본인이 관심 있는 분야의 데이터를 사용하면 됩니다. 예시 주제와 데이터셋 목록을 제공합니다.

\subsection{과제 및 평가}

\textbf{Q4: 숙제에서 협업이 가능한가요?}

A: 토론은 가능하지만, 코드는 각자 작성해야 합니다. 동일한 코드 제출 시 학사 경고 대상입니다.

\textbf{Q5: Late Days는 어떻게 사용하나요?}

A: 별도 신청 없이 자동으로 적용됩니다. 마감일 이후 제출하면 Late Day가 차감됩니다.

\textbf{Q6: 출석을 놓치면 어떻게 되나요?}

A: 강의 녹화를 시청하고 퀴즈에 응시하면 출석으로 인정됩니다. 단, 출석률에 따라 최종 성적에 상한이 적용됩니다.

\subsection{기술적 질문}

\textbf{Q7: 어떤 컴퓨터가 필요한가요?}

A: Python과 Jupyter Notebook을 실행할 수 있는 노트북이면 충분합니다. Google Colab(무료)도 사용 가능합니다.

\textbf{Q8: 데이터셋은 어디서 구하나요?}

A: Kaggle, UCI ML Repository, government open data 등을 활용할 수 있습니다. 강의에서 추천 목록을 제공합니다.

\section{학습 전략 및 조언 (Study Strategies)}

\subsection{효과적인 학습 방법}

\textbf{1. 실습 중심 학습:}
\begin{itemize}
    \item 강의를 들은 후 반드시 코드를 직접 작성
    \item 예제 데이터를 변형하여 실험
    \item 에러 메시지를 두려워하지 말고 디버깅 연습
\end{itemize}

\textbf{2. 점진적 이해:}
\begin{itemize}
    \item 한 번에 모든 것을 이해하려 하지 말 것
    \item 기본 개념부터 차근차근 쌓아가기
    \item 이해가 안 되면 질문하기 (Ed Discussion)
\end{itemize}

\textbf{3. 프로젝트 기반 학습:}
\begin{itemize}
    \item 관심 있는 분야의 데이터로 연습
    \item 작은 프로젝트부터 시작하여 확장
    \item GitHub에 포트폴리오 구축
\end{itemize}

\subsection{시간 관리}

\textbf{주간 학습 계획 (권장):}
\begin{itemize}
    \item 강의 시청: 3-4시간
    \item 실습 및 복습: 2-3시간
    \item 숙제: 4-6시간
    \item 프로젝트: 2-3시간
    \item \textbf{총 주당 약 12-15시간}
\end{itemize}

\textbf{과제 관리 팁:}
\begin{itemize}
    \item 마감일 최소 3일 전에 시작
    \item Late Days는 비상용으로 보관
    \item 막히면 조기에 도움 요청
\end{itemize}

\section{핵심 개념 정리 (Key Concepts Summary)}

\subsection{데이터 과학의 핵심 원칙}

\begin{enumerate}
    \item \textbf{질문 중심 접근 (Question-Driven Approach)}
    \begin{itemize}
        \item 명확한 목표 설정
        \item 측정 가능한 성과 지표
    \end{itemize}

    \item \textbf{데이터 품질 우선 (Data Quality First)}
    \begin{itemize}
        \item Garbage in, garbage out
        \item 철저한 데이터 정제 및 검증
    \end{itemize}

    \item \textbf{반복적 개선 (Iterative Improvement)}
    \begin{itemize}
        \item 프로토타입 → 평가 → 개선
        \item 지속적인 모델 업데이트
    \end{itemize}

    \item \textbf{해석 가능성 (Interpretability)}
    \begin{itemize}
        \item 블랙박스 모델 지양
        \item 결과의 근거 제시
    \end{itemize}

    \item \textbf{윤리적 고려 (Ethical Considerations)}
    \begin{itemize}
        \item 편향성 인식 및 완화
        \item 개인정보 보호
        \item 공정성 및 투명성
    \end{itemize}
\end{enumerate}

\subsection{수식 정리 (Formula Reference)}

\textbf{회귀 분석 (Regression Analysis):}

단순 선형 회귀:
\[
y = \beta_0 + \beta_1 x + \epsilon
\]

다중 선형 회귀:
\[
y = \beta_0 + \beta_1 x_1 + \beta_2 x_2 + \cdots + \beta_p x_p + \epsilon
\]

정규 방정식 (Normal Equation):
\[
\hat{\beta} = (X^T X)^{-1} X^T y
\]

\textbf{모델 평가 지표:}

평균 절대 오차 (MAE):
\[
MAE = \frac{1}{n}\sum_{i=1}^{n} |y_i - \hat{y}_i|
\]

평균 제곱근 오차 (RMSE):
\[
RMSE = \sqrt{\frac{1}{n}\sum_{i=1}^{n} (y_i - \hat{y}_i)^2}
\]

결정 계수 ($R^2$):
\[
R^2 = 1 - \frac{\sum_{i=1}^{n}(y_i - \hat{y}_i)^2}{\sum_{i=1}^{n}(y_i - \bar{y})^2}
\]

\section{다음 강의 예고 (Next Lecture Preview)}

\subsection{Lecture 2: 데이터 수집 및 처리}

다음 강의에서 다룰 내용:

\begin{itemize}
    \item \textbf{웹 스크래핑 (Web Scraping):}
    \begin{itemize}
        \item BeautifulSoup 사용법
        \item HTML 파싱
        \item 윤리적 스크래핑
    \end{itemize}

    \item \textbf{pandas 기초:}
    \begin{itemize}
        \item DataFrame 생성 및 조작
        \item 데이터 필터링 및 정렬
        \item 결측치 처리
    \end{itemize}

    \item \textbf{데이터 정제 (Data Cleaning):}
    \begin{itemize}
        \item 이상치 탐지 및 처리
        \item 데이터 타입 변환
        \item 중복 제거
    \end{itemize}
\end{itemize}

\subsection{준비 사항}

다음 강의 전에 완료할 것:
\begin{enumerate}
    \item Python 및 Jupyter Notebook 설치 확인
    \item pandas, BeautifulSoup 라이브러리 설치
    \item Homework 0 (Python 복습) 완료
    \item Ed Discussion에 가입
\end{enumerate}

\section{추가 학습 자료 (Additional Resources)}

\subsection{추천 교재}

\begin{enumerate}
    \item \textbf{Python for Data Analysis} by Wes McKinney
    \begin{itemize}
        \item pandas 라이브러리 공식 가이드
        \item 데이터 조작 및 분석 기법
    \end{itemize}

    \item \textbf{Introduction to Statistical Learning} by James, Witten, Hastie, Tibshirani
    \begin{itemize}
        \item 통계 학습의 기초
        \item R 코드 예제 (Python으로 변환 가능)
    \end{itemize}

    \item \textbf{Hands-On Machine Learning} by Aurélien Géron
    \begin{itemize}
        \item scikit-learn 활용
        \item 실전 머신러닝 프로젝트
    \end{itemize}
\end{enumerate}

\subsection{온라인 자료}

\textbf{공식 문서:}
\begin{itemize}
    \item Python: \url{https://docs.python.org/3/}
    \item pandas: \url{https://pandas.pydata.org/docs/}
    \item scikit-learn: \url{https://scikit-learn.org/}
    \item matplotlib: \url{https://matplotlib.org/}
\end{itemize}

\textbf{연습 플랫폼:}
\begin{itemize}
    \item Kaggle: 데이터셋 및 경진대회
    \item DataCamp: 인터랙티브 Python 학습
    \item LeetCode: 알고리즘 문제 풀이
\end{itemize}

\section{결론 (Conclusion)}

CS109A는 데이터 과학의 전체 파이프라인을 학습하는 종합적인 과정입니다. 이 강의를 통해:

\begin{itemize}
    \item 실제 데이터를 수집하고 처리하는 능력 배양
    \item 통계적 모델링 및 머신러닝 기법 습득
    \item Python을 활용한 데이터 분석 도구 활용
    \item 결과를 효과적으로 시각화하고 전달하는 방법 학습
    \item 팀 프로젝트를 통한 실전 경험 축적
\end{itemize}

\textbf{성공을 위한 핵심 요소:}
\begin{enumerate}
    \item 꾸준한 실습과 코딩 연습
    \item 적극적인 질문 및 토론 참여
    \item 프로젝트에 관심 있는 분야 적용
    \item 출석 및 과제 마감일 준수
    \item Teaching Fellows 및 동료들과의 협력
\end{enumerate}

데이터 과학은 지속적으로 발전하는 분야입니다. 이 강의는 여러분의 데이터 과학 여정의 시작점이 될 것입니다.

\vspace{1cm}

\textbf{행운을 빕니다! Good luck!}

\end{document}
